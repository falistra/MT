\documentclass[a4paper]{article}
% AMS Packages
\usepackage{amsmath}
\usepackage{amsthm}
\usepackage{amssymb}
\usepackage[pages=all, color=black, position={current page.south}, placement=bottom, scale=1, opacity=1, vshift=5mm]{background}
\SetBgContents{
	$Corso\ di\ Informatica\ Teorica\ -\ Universit\grave{a}\ di\ Bologna$
}      % copyright

\usepackage[margin=1in]{geometry} % full-width



% Unicode
\usepackage[utf8]{inputenc}
\usepackage{hyperref}
\hypersetup{
	unicode,
%	colorlinks,
%	breaklinks,
%	urlcolor=cyan, 
%	linkcolor=blue, 
	pdfauthor={Fabio Zanasi},
	pdftitle={La macchina di Turing Universale},
	pdfsubject={Descrizione dettagliata dell'implementazione},
	pdfkeywords={Turing, Macchina, Universale, Implementazione},
	pdfproducer={LaTeX},
	pdfcreator={pdflatex}
}

% Vietnamese
%\usepackage{vntex}

% Natbib
\usepackage[sort&compress,numbers,square]{natbib}
\bibliographystyle{mplainnat}

% Theorem, Lemma, etc
\theoremstyle{plain}
\newtheorem{theorem}{Theorem}
\newtheorem{corollary}[theorem]{Corollary}
\newtheorem{lemma}[theorem]{Lemma}
\newtheorem{claim}{Claim}[theorem]
\newtheorem{axiom}[theorem]{Axiom}
\newtheorem{conjecture}[theorem]{Conjecture}
\newtheorem{fact}[theorem]{Fact}
\newtheorem{hypothesis}[theorem]{Hypothesis}
\newtheorem{assumption}[theorem]{Assumption}
\newtheorem{proposition}[theorem]{Proposition}
\newtheorem{criterion}[theorem]{Criterion}
\theoremstyle{definition}
\newtheorem{definition}[theorem]{Definition}
\newtheorem{example}[theorem]{Example}
\newtheorem{remark}[theorem]{Remark}
\newtheorem{problem}[theorem]{Problem}
\newtheorem{principle}[theorem]{Principle}

\usepackage{graphicx, color}
\graphicspath{{fig/}}

%\usepackage[linesnumbered,ruled,vlined,commentsnumbered]{algorithm2e} % use algorithm2e for typesetting algorithms
\usepackage{algorithm, algpseudocode} % use algorithm and algorithmicx for typesetting algorithms
\usepackage{mathrsfs} % for \mathscr command

\usepackage{lipsum}

% Author info
\title{La macchina di Turing Universale}
\author{Fabio Zanasi$^1$}

\date{
	$^1$ $Universit\grave{a}\ di\ Bologna$ 
}

\begin{document}
	\maketitle
	\begin{abstract}
		Una Macchina di Turing Universale $\grave{e}$ in grado di emulare il comportamento di una qualsiasi
		macchina di Turing codificata sul suo nastro.   
		
		\noindent\textbf{Keywords:} UTM, MTU, Turing
	\end{abstract}

	\tableofcontents
	
	\section{Descrizione}
	\label{sec:intro}
	
	\subsection{Struttura}
	\label{sec:pre}
	
La macchina di Turing universale ($\mathfrak{U}$) $\grave{e} $ in grado di emulare una 
qualsiasi macchina di Turing ($\mathfrak{M}$) il cui insieme di simboli consiste di tre soli simboli : 
\textvisiblespace (blank), $\mathbf{0}$ e $\mathbf{1}$.

La configurazione corrente della macchina ($\mathfrak{M}$)  $\grave{e}$ codificata sul
nastro di ($\mathfrak{U}$) come segue:
\vspace{5pt}
\begin{center}
\hrule
\vspace{2pt}
\ldots\ \textvisiblespace\ \textvisiblespace\ \textit{sinistra-nastro}\ [\ $\Hat{\mathfrak{M}}$ ]\ $\boxdot$\ \textit{destra-nastro}\ \textvisiblespace\ \textvisiblespace\ \ldots 
\vspace{2pt}
\hrule
\end{center}
\vspace{5pt}

\begin{itemize}
	\item \textit{sinistra-nastro}\ $\grave{e}$ il contenuto del nastro di $\mathfrak{M}$ alla sinistra della cella corrente.
	\item $\Hat{\mathfrak{M}}$ $\grave{e}$ la codifica (notare $\hat{}$\ !) delle regole di transizione per $\mathfrak{M}$ , ove $\grave{e}$ segnato lo stato corrente.
	\item $\boxdot$\ $\grave{e}$ il contenuto della cella corrente di $\mathfrak{M}$.
	\item \textit{destra-nastro}\ $\grave{e}$ il contenuto del nastro di $\mathfrak{M}$ alla destra della cella corrente.
  \end{itemize}
	
	\subsection{La codifica delle regole di transizione per $\mathfrak{M}$ }
	\label{sec:prev-results}
	
	Gli stati di $\mathfrak{M}$ sono numerati $q_1$ , \ldots , $q_m$. Nel seguito denotiamo con 
	$\Hat{q_i}$ la codifica dello stato $q_1$ (notare $\hat{}$\ !) . Se $q_C$ $\grave{e}$ lo stato corrente, 
	La codifica delle regole di transizione per $\mathfrak{M}$ ha la forma:

	\vspace{3pt}
	\begin{center}
		$\Hat{q_1}$\ :\ $\Hat{q_2}$\ :\ \ldots\ $\Hat{q}_{C-1}$\ :\ $\Hat{q}_{C}$\ !\ $\Hat{q}_{C+1}$\ :\ \ldots\ $\Hat{q}_{m}$  
	\end{center}
	\vspace{3pt}

	dove $\Hat{q_i}$\ codifica (notare $\hat{}$\ !)  la regola di transizione per lo stato $\Hat{q_i}$\ , la codifica di ogni stato termina con il i due punti (:) 
	eccetto che per lo stato corrente che $\grave{e}$ individuato per terminare con un punto esclamativo.
	\\
	Data la regola di transizione per lo stato $\Hat{q_i}$\ :

	\vspace{3pt}
		\begin{itemize}
			\item \begin{center}
			$q_i$,\textvisiblespace\ $\rightarrow$\ $s_j$,$D_j$,$q_j$
			\end{center}
			\item \begin{center}
			$q_i$,0\ $\rightarrow$\ $s_k$,$D_k$,$q_k$
			\end{center}			
			\item  \begin{center}
			$q_i$,1\ $\rightarrow$\ $s_l$,$D_l$,$q_l$
			\end{center}	
		\end{itemize}	
	\vspace{3pt}

	dove $s_j$,$s_k$,$s_l$ $\in$ $\{$ \textvisiblespace,0,1$\}$ , $D_j$,$D_k$,$D_k$ $\}$  $\in$ $\{$ $\mathbf{L}$,$\mathbf{R}$ $\}$ ( la testina si muove $\mathbf{sempre}$ ) 
	e 1 $\leqslant$ j,k,l $\leqslant$ m (numero degli stati), allora
	\\
	la codifica $\Hat{q_i}$\  ha la forma

	\vspace{3pt}
	\begin{center}
		$s_j$\ $D_j$\ $\sigma_{j-i}$\ ,\ $s_k$\ $D_k$\ $\sigma_{k-i}$\ ,\ $s_l$\ $D_l$\ $\sigma_{l-i}$\ 
	\end{center}
	\vspace{3pt}

	dove le tre parti della codifica della regola di transizione sono separate dalla virgola (,) e

	\vspace{3pt}
	\begin{center}

		$\sigma_{\delta}$ = $\left\{ \begin{array}{rcl} . \ \ se\ \delta = 0, \\ +^{\delta} \ \ se\ \delta > 0, \\ -^{|\delta|} \ \ se\ \delta < 0, \end{array}\right. $

	\end{center}
	\vspace{3pt}

	gli esponenti (p.e. $\delta$ in $+^{\delta}$) denotano ripetizione. 
	Cos$\grave{i}$ una sequenza di $|\delta|$ simboli (+) o (-) codificano il relativo cambiamento nel numero di stato, il punto (.) 
	denota nessun cambiamento di stato.
	Se non c'$\grave{e}$ una regola di transizione per (${q_i}$,s) (interpretata come una configurazione di arresto), allora la codifica
	$\Hat{q_i}$\ $\grave{e}$ la stringa vuota. 
	\\
	La dimensione di $\Hat{\mathfrak{M}}$ $\grave{e}$ O($m^2$), dove $m$ $\grave{e}$ il numero di stati di $\mathfrak{M}$.
	
	\subsection{Esempio}
	\label{sec:examples}
	La macchina di Turing $\mathfrak{M}$ a due stati per incrementare un numero in formato binario :

	\vspace{3pt}
		\begin{itemize}
			\item \begin{center}
			$q_1$,\textvisiblespace\ $\rightarrow$\ $\mathbf{\textvisiblespace}$,$\mathbf{L}$,$q_2$
			\end{center}
			\item  \begin{center}
			$q_1$,0\ $\rightarrow$\ $\mathbf{0}$,$\mathbf{R}$,$q_1$	
			\end{center}		
			\item \begin{center}
			$q_1$,1\ $\rightarrow$\ $\mathbf{1}$,$\mathbf{R}$,$q_1$
			\end{center}			
		\end{itemize}	
	\vspace{1pt}

	\vspace{1pt}
		\begin{itemize}
			\item  \begin{center}
			$q_2$,\textvisiblespace\ $\rightarrow$\ $\mathbf{1}$,$\mathbf{L}$,$q_3$
			\end{center}			
			\item \begin{center}
			$q_2$,0\ $\rightarrow$\ $\mathbf{1}$,$\mathbf{L}$,$q_3$
			\end{center}			
			\item \begin{center}
			$q_2$,1\ $\rightarrow$\ $\mathbf{0}$,$\mathbf{L}$,$q_2$
			\end{center}
		\end{itemize}	
	\vspace{3pt}

	\vspace{1pt}
		\begin{itemize}
			\item  \begin{center}
			$q_3$,\textvisiblespace\ $\rightarrow$\ $\mathbf{\textvisiblespace\ }$,$\mathbf{L}$,$halt$
			\end{center}			
			\item \begin{center}
			$q_3$,0\ $\rightarrow$\ $\mathbf{0}$,$\mathbf{L}$,$q_3$
			\end{center}			
			\item \begin{center}
			$q_3$,1\ $\rightarrow$\ $\mathbf{1}$,$\mathbf{L}$,$q_3$
			\end{center}
		\end{itemize}	
	\vspace{3pt}


con stato iniziale $q_1$, input iniziale $\mathbf{1011}$, ha questa codifica $\Hat{\mathfrak{M}}$

\vspace{3pt}
\begin{center}
$\mathbf{[\ L+,0R.,1R.!1L+,1L+,0L.:,0L.,1L.:]1011}$
\end{center}
\vspace{3pt}

\section{Implementazione}
\label{sec:Impl}


\subsection{Insieme dei simboli}
\label{sec:SymbolSet}

$\mathfrak{U}$ fa uso di un insieme di 16 simboli 
$\{$ $\mathbf{\textvisiblespace}$  $\mathbf{0}$  $\mathbf{1}$ $\mathbf{[}$ 
$\mathbf{]}$ $\mathbf{,}$ $\mathbf{:}$ $\mathbf{!}$ $\mathbf{L}$ $\mathbf{R}$
$\mathbf{.}$ $\mathbf{-}$ $\mathbf{+}$ $\mathbf{\#}$ $\mathbf{<}$ $\mathbf{>}$
$\}$ 

\subsection{Algoritmo}
\label{sec:Impl}

$\mathfrak{U}$ inizia posizionandosi alla prima cella a destra di $\mathbf{]}$ , ove si trova il primo
simbolo di input di $\mathfrak{M}$.

\begin{enumerate}
	\item Legge il simbolo, $s$, dalla cella corrente.
	\item Trova lo stato corrente, muovendosi a sinistra fino a quando non incontra il 
	simbolo $\mathbf{!}$
	\item Trova la regola di transizione corrente muovendosi a sinistra oltre 2, 1 o 0 virgole (,)
	se $s$ $\grave{e}$ $\mathbf{\textvisiblespace}$  $\mathbf{0}$  $\mathbf{1}$ rispettivamente.
	\item Se la regola corrente $\grave{e}$ vuota , $\mathfrak{U}$ si ferma.
	\item Altrimenti, modifica la codifica dello stato corrente, 
	cambiando il punto $\mathbf{.}$ in $\mathbf{\#}$, ogni $\mathbf{-}$ in $\mathbf{<}$ e
	ogni $\mathbf{+}$ in $\mathbf{<}$
	\item Legge il simbolo da scrivere dalla regola corrente , 
	si muove verso destra e lo scrive nella cella corrente di $\mathfrak{M}$
	\item Muove a sinistra alla regola corrente (riconoscibile dalla presenza di $\mathbf{\#}$, 
	$\mathbf{<}$ $\mathbf{>}$ ) e legge la direzione di spostamento ($\mathbf{L}$ o $\mathbf{R}$)
	\item Sposta l'intera codifica di $\mathfrak{M}$ a sinistra o a destra di una cella, 
	poi aggiorna la cella corrente di $\mathfrak{M}$.
	\item Cambia lo stato se necessario sostituendo ogni $\mathbf{<}$ con $\mathbf{-}$ o ogni
	$\mathbf{>}$ con $\mathbf{+}$, e ogni volta , muovendo il simbolo $\mathbf{!}$ alla
	posizione del successivo $\mathbf{:}$ alla sinistra o alla destra rispettivamente. 
	Se non $\grave{e}$ richiesto un cambiamento di stato, il simbolo $\mathbf{\#}$ $\grave{e}$ 
	sostituito con un $\mathbf{.}$
	\item Muove a destra alla cella corrente di $\mathfrak{M}$ per ricominciare dal passo 1.

\end{enumerate}

%	\newpage
	\bibliography{refs}
	
\end{document}